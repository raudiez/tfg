\section{Disposiciones legales y Normas aplicadas}

\begin{itemize}
    \item \textbf{UNE 157801:2007} - Criterios generales para la elaboración de proyectos de sistemas de información.
    \item \textbf{UNE 157001:2014} - Criterios generales para la elaboración formal de los documentos que constituyen un proyecto técnico.
\end{itemize}

    
%% Bibliografía como section.
\nocite{*}
\bibliographystyle{plain}
\bibliography{main}

\section{Metodología de desarrollo}

En este Proyecto se ha decidido optar por aplicar lo que se conocen como metodologías ágiles, siendo este tipo de metodologías las que se basan en la división del trabajo en determinadas secciones funcionales o etapas que se van revisando e incrementando, con el objetivo de detectar errores y corregirlos poco a poco, llegando a una solución del Proyecto más rápidamente que con las metodologías de desarrollo clásicas. \\

Dentro de las metodologías ágiles se ha decidido utilizar la denominada SCRUM, metodología que permite seguir un proceso incremental basado en la división y el desarrollo de funcionalidades que se validan y corrigen conforme se avanza en la realización del Proyecto. \\

Esto proporciona cierta flexibilidad en el desarrollo frente a cualquier tipo de cambio, ya que se pueden tratar como cambios menores. Si se compara esta con otro tipo de metodología, se puede apreciar que es una gran ventaja, ya que en otras metodologías en las que hay que desarrollar una versión de prototipo o una versión totalmente operativa a partir de la cual se corregirá lo que sea necesario, cualquier modificación puede suponer un sobre-coste o un gasto de tiempo con el que al seguir la metodología SCRUM se ahorra. \\

Otra gran ventaja de esta metodología es que el resultado suele ser de mayor calidad, ya que al abordar funcionalidades poco a poco, y no la totalidad funcional del producto final, el desarrollo se lleva a cabo de una forma más cuidadosa, y es más fácil evitar fallos en el producto final. \\

Esta metodología, SCRUM, se ha aplicado durante toda la realización del Proyecto, ya sea para el diseño del sistema, el software del nodo central, la aplicación para smartphone o incluso el diseño de los sistemas de interconexión creados.

\section{Herramientas}

    \subsection{Herramientas software}
    
    Android Studio - Software para el desarrollo de aplicaciones Android proporcionado por Google, utilizado para la programación de la aplicación desarrollada en este Proyecto. \\
    
    Eagle - Software de creación de circuitos electrónicos para su implementación en placas de circuito impreso. Utilizado para crear las PCB utilizadas en el Proyecto. \\
    
    Electrodroid - Aplicación para smartphone que contiene una colección de herramientas software electrónicas y documentación técnica, como calculadoras de ancho de pistas para PCB, de caída de voltaje y otras herramientas útiles. \\
    
    LPKF Circuit Pro - Software propio de la fresadora para PCB utilizada, usado para la fabricación de las placas de circuito impreso del Proyecto. \\
    
    NinjaMock - Herramienta online para la creación de mockups o prototipos de diseño de aplicaciones, empleado en la realización del prototipado de la aplicación. \\
    
    Python - Lenguaje de programación en el que se basa el sistema principal de este Proyecto, cada vez más extendido. \\
    
    Sublime Text - Editor de texto especialmente diseñado para código, con multitud de complementos para auto-generar código rápidamente.
    
    \subsection{Herramientas mecánicas}
    
    Fresadora PCB LPKF ProtoMat S103 - Utilizada para obtener las placas de circuito impreso utilizadas en el Proyecto.
    
\section{Otras referencias}

    \subsection{Repositorio en GitHub}
    \label{sec:repogithub}
    
    Tanto la documentación como el código fuente asociado descrito en este documento, están alojados en GitHub, disponibles en \url{https://github.com/raudiez/tfg}.