Actualmente existen varias soluciones a los problemas que pueda tener una persona con discapacidad auditiva, independientes del grado de discapacidad que padezca.\\
\\
Una de las soluciones consiste en un aparato de bolsillo de avisos por señales de vibración y luces de colores~\cite{visiofarma}, que recibe los avisos de distintos aparatos receptores de ruido que pueden obtener información del ruido generado por el timbre, el teléfono, el portero automático...\\
\\
Esta solución requeriría que el discapacitado lleve consigo siempre un aparato en el bolsillo, lo cual puede ser un poco molesto. Además, tiene una limitación de número de alarmas a captar.\\
\\
Otra posible solución es la instalación de ventanas en la mayoría de las paredes de la vivienda (paredes interiores), que permitiría ver las alertas visuales desde cualquier habitación. Este sistema necesitaría que previamente se modificasen los aparatos que emiten alertas sonoras para que emitan alertas visuales, además de reducir la intimidad de las personas que habiten en la vivienda.\\
\\
Una solución diferente a las anteriores es adaptar la instalación eléctrica de la vivienda~\cite{avisossordos}, instalando tantas lámparas por habitación como alertas visuales se quieran mostrar. Estas lámparas mostrarían alertas visuales para el teléfono, el timbre, etc. La mayor desventaja aquí es el consumo de bombillas necesarias para una instalación completa, y el espacio que estas ocuparían.