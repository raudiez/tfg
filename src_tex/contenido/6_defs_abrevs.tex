\begin{description}
    \item[\textbf{AAL}] Ambient Assisted Living (Vida Asistida en la Vivienda).
    \item[\textbf{Arduino}] Plataforma electrónica de hardware libre de fácil uso, utilizada normalmente para el montaje de dispositivos multifuncionales.
    \item[\textbf{AWG}] American Wire Gauge, es una escala de medidas o calibres para conductores eléctricos adoptada como estándar. Es necesario utilizar una tabla de equivalencias~\cite{awg_table} para encontrar la correspondencia en diámetro de cable entre AWG y mm.
    \item[\textbf{Bluetooth}] Especificación industrial para WPAN que posibilita la transmisión de datos entre diferentes dispositivos con corto alcance. 
    \item[\textbf{Domótica}] Conjunto de sistemas capaces de automatizar una vivienda.
    \item[\textbf{Ethernet}] Estándar de red local para computadores (IEEE 802.3).
    \item[\textbf{GPIO}] General Purpose Input/Output (Entrada/Salida de Propósito General).
    \item[\textbf{IDE}] Integrated Development Environment, o Ambiente de Desarrollo Integrado, es una herramienta software que proporciona servicios integrales en el desarrollo software.
    \item[\textbf{IEEE}] Institute of Electrical and Electronics Engineers (Instituto de Ingeniería Eléctrica y Electrónica).
    \item[\textbf{JSON}] JavaScript Object Notation, es un formato de texto ligero para el intercambio de datos, muy simple y extendido, y representa la estructura de los datos en el propio texto.
    \item[\textbf{Metodología}] Una metodología es el conjunto de pasos, procedimientos, técnicas, herramientas y el soporte documental que ayudan a desarrollar un producto.
    \item[\textbf{LAN}] Local Area Network (Red de Área Local).
    \item[\textbf{PAN}] Personal Area Network (Red de Área Personal).
    \item[\textbf{PCB}] Printed Circuit Board (Placa de Circuito Impreso).
    \item[\textbf{Piconet}] Red informática cuyos nodos se conectan utilizando Bluetooth. Puede constar de 2 a 7 dispositivos.
    \item[\textbf{Raspberry Pi}] Computador embebido de bajo coste desarrollado en Reino Unido por la Raspberry Pi Foundation.
    \item[\textbf{REST}] Representational State Transfer, es un estilo de arquitectura software para sistemas hipermedia distribuidos, y describe cualquier interfaz entre sistemas que utilice directamente métodos HTTP (GET, POST, PUT y DELETE) realizar operaciones con datos.
    \item[\textbf{WEP}] Wired Equivalent Privacy (Privacidad Equivalente a Cableado).
    \item[\textbf{WiFi}] WiFi (o Wi-Fi) es el nombre de una marca comercial de Wi-Fi Alliance, definido como cualquier producto basado en conexión WLAN que cumple con el estándar IEEE 802.11. Comúnmente se confunde con ese estándar, IEEE 802.11.
    \item[\textbf{WLAN}] Wireless Local Area Network (Red de Área Local Inalámbrica).
    \item[\textbf{WPA}] Wi-Fi Protected Access (Acceso Wi-Fi Protegido).
    \item[\textbf{WPAN}] Wireless Personal Area Network (Red de Área Personal Inalámbrica).
\end{description}